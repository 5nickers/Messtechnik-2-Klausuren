\input{header.tex}


\begin{document}

\maketitle

Dieser Text ist unter dieser \href{http://creativecommons.org/licenses/by-nc-sa/4.0/}{Creative Commons} Lizenz veröffentlicht.

\textcolor{red}{Ich erhebe keinen Anspruch auf Vollständigkeit oder Richtigkeit. Falls ihr Fehler findet oder etwas fehlt, dann meldet euch bitte über den Emailkontakt.}

\tableofcontents


\newpage



\section{Aufgabe 1}

Die richtigen Antworten sind:


\begin{center}
	\begin{tabular}{c|c|c|c|c|c|c|c}
		1 & 2 & 3 & 4 & 5 & 6 & 7 & 8 \\ 
		\hline d & a & d & c & a,d & c & / & c \\  
	\end{tabular} 
\end{center}


\section{Aufgabe 2}

\subsection*{a)}

Wir haben eine Frequenz von $f = \frac{1}{\unit[5]{ms}} = \unit[200]{Hz}$

\subsection*{b)}

Die effektive Spannung können wir dann nach folgender Formel bestimmen:

\begin{align*}
	U^2 &= \frac{1}{5} \cdot \left[ 4 \cdot 20^2 + 1 \cdot \left( 40 \right)^2 \right] = \unit[3200]{\mu V^2} \\
	\hfill \\
	\Rightarrow U &= \unit[56,56]{\mu V}
\end{align*}


\subsection*{c)}

Das Gleichstrommessgerät müsste $U = \frac{4 \cdot 20 - 40}{5} = \unit[8]{\mu V}$ anzeigen.


\subsection*{d)}

Das Wechselstrommessgerät zeigt nur den Wechselanteil an:

\paragraph{Schritt 1}

Wechselspannungseinkopplung, das macht den Gleichtaktanteil zu 0.

\paragraph{Schritt 2}

Gleichrichtung

\paragraph{Schritt 3}

arithmetischer Mittelwert $\unit[24]{\mu A}$

\paragraph{Schritt 4}

Multiplikation mit Formfaktor für Sinus

\begin{align*}
U_\sim = 1,11111 \cdot 24 = \unit[26,66]{\mu V}
\end{align*}


\section{Aufgabe 3}

\subsection*{a)}


\begin{figure}[h]
	\centering
	\includegraphics[scale=0.15]{A3_1.jpg}
\end{figure}


\subsection*{b)}

\begin{align*}
I_1 &= \frac{U_0}{R + \Delta R - \Delta R + R} = \frac{U_0}{2R} = I_2
\intertext{Wir betrachten nun die Masche I:}
0 &= I(R - \Delta R) + U_B - I(R + \Delta R) \\
\Leftrightarrow 0 &= \frac{U_0}{2R} \cdot R - \frac{U_0}{2R} \cdot \Delta R + U_B - \frac{U_0}{2R} \cdot R - \frac{U_0}{2R} \cdot \Delta R \\
\Leftrightarrow U_B &= \frac{\Delta R}{R} \cdot U_0
\end{align*}


\subsection*{c)}

\begin{align*}
\frac{\Delta R}{R} &= k \cdot \epsilon \\
\Leftrightarrow U_B &= k \cdot \epsilon \cdot U_O 
\intertext{Für die Empfindlichkeit müssen wir nach $\epsilon$ ableiten:}
E &= \frac{\p U_B}{\p \epsilon} = k \cdot U_0 = 2 \cdot 24 = \unit[48]{V}
\end{align*}


\section{Aufgabe 4}

\subsection*{a)}

\begin{figure}[h]
	\centering
	\includegraphics[scale=0.2]{A4_1.jpg}
	\caption{$U_1 = \unit[51,875]{mV}, U_2 = \unit[900]{mV}, R_1 = \unit[10]{k \Omega}$}	
\end{figure}

Mir ist allerdings nicht klar was man mit dem Biasstrom macht.

\subsection*{b)}

Zuerst bestimmen wir den Widerstand $R_2$:

\begin{align*}
\frac{R_2}{R_1} &= \frac{U_2}{U_1} \\
\Leftrightarrow R_2 &= \frac{900}{51,875} \cdot 10 = \unit[173,5]{k \Omega}
\intertext{Wir wissen, das bei der Grenzfrequenz die Verstärkung 1 ist. Darüber können wir nun die Kapazität $C_2$ ausrechnen:}
V &= \frac{R_2}{R_1} \cdot \frac{1}{1 + j \omega R_2 C_2} = 1 \\
\Leftrightarrow \frac{R_2}{R_1} - 1 &= j \omega R_2 C_2 \\
\Leftrightarrow C_2 &= \frac{\frac{R_2}{R_1} - 1}{\omega R_2} = \frac{\frac{173500}{10000} - 1}{2\pi \cdot 0,5 \cdot 173500} = \unit[30]{\mu F}
\end{align*}

\subsection*{c)}

Wir setzen in die Formel von b) ein:

\begin{align*}
V &= \frac{R_2}{R_1} \cdot \frac{1}{1 + j \omega R_2 C_2} = \frac{173500}{10000} \cdot \frac{1}{1 + 2 \pi \cdot 50 \cdot 173500 \cdot 30 \cdot 10^{-6}} \approx 1,01
\end{align*}





\end{document}