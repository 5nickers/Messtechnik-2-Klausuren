\input{header.tex}


\begin{document}

\maketitle

Dieser Text ist unter dieser \href{http://creativecommons.org/licenses/by-nc-sa/4.0/}{Creative Commons} Lizenz veröffentlicht.

\textcolor{red}{Ich erhebe keinen Anspruch auf Vollständigkeit oder Richtigkeit. Falls ihr Fehler findet oder etwas fehlt, dann meldet euch bitte über den Emailkontakt.}

\tableofcontents


\newpage



\section{Aufgabe 1}


Die richtigen Antworten sind:


\begin{center}
	\begin{tabular}{c|c|c|c|c|c|c|c}
				 1 & 2 & 3 & 4 & 5 & 6 & 7 & 8 \\ 
		\hline b,c & a & c & d & b,d & c & b & b,d \\  
	\end{tabular} 
\end{center}


\section{Aufgabe 2}

Wir haben eine Frequenz von $f = \frac{1}{\unit[4]{ms}} = \unit[250]{Hz}$

\subsection*{a)}

Die effektive Spannung können wir dann nach folgender Formel bestimmen:

\begin{align*}
U &= \sqrt{\frac{1}{T} \int_{0}^{T} U^2 \d t}
\intertext{Aus Symetriegründen reicht es hier über die halbe Periode zu integrieren:}
U(t) &= \unit[1]{V/ms} \cdot t \\
U^2 &= \frac{1}{2} \int_{0}^{\unit[2]{ms}} t^2 \d t = \frac{1}{2} \left[ \frac{1}{3} t^3 \right]_0^2  = \half \cdot \frac{1}{3} \cdot 8 = \unit[1,333]{V^2} \\
\hfill \\
\Rightarrow U &= \unit[1,155]{V}
\end{align*}

\subsection*{b)}

Das es eine reine Wechselgröße ist, zeigt das Messgerät $\unit[0]{V}$ an.


\subsection*{c)}


\paragraph{Schritt 1}

Wechselspannungseinkopplung

\paragraph{Schritt 2}

Gleichrichtung

\paragraph{Schritt 3}

arithmetischer Mittelwert

\begin{align*}
\bar{|U|} = \frac{1}{T} \int U \d t = \half \int_{0}^{2} 1 - t \d t = \left. \half t^2 \right|_0^2 = \unit[1]{V}
\end{align*}

\paragraph{Schritt 4}

Multiplikation mit Formfaktor für Sinus

\begin{align*}
U_\sim = 1,11111 \cdot 1 = \unit[1,1111]{V}
\end{align*}


\section{Aufgabe 3}

\subsection*{a)}


Bild einfügen!!


\subsection*{b)}

\begin{align*}
I_1 &= \frac{U_0}{R + \Delta R - \Delta R + R} = \frac{U_0}{2R} = I_2
\intertext{Wir betrachten nun die Masche I:}
0 &= I(R - \Delta R) + U_B - I(R + \Delta R) \\
\Leftrightarrow 0 &= \frac{U_0}{2R} \cdot R - \frac{U_0}{2R} \cdot \Delta R + U_B - \frac{U_0}{2R} \cdot R - \frac{U_0}{2R} \cdot \Delta R \\
\Leftrightarrow U_B &= \frac{\Delta R}{R} \cdot U_0
\end{align*}


\subsection*{c)}

\begin{align*}
\frac{\Delta R}{R} &= k \cdot \epsilon \\
\Leftrightarrow U_B &= k \cdot \epsilon \cdot U_O 
\intertext{Für die Empfindlichkeit müssen wir nach $\epsilon$ ableiten:}
E &= \frac{\p U_B}{\p \epsilon} = k \cdot U_0 = 2 \cdot 24 = \unit[48]{V}
\end{align*}


\section{Aufgabe 4}

\subsection*{a)}

Der Operationsverstärker ist invertierend , weil die Eingangsgröße auf $-$ liegt.

\subsection*{b)}

Es handelt sich um einen Integrierer (Kondensator) und Addierer (Anordnung der Eingangsspannungen)


\subsection*{c)}

\begin{align*}
I &= I_1 + I_2 \\
I_1 &= \frac{U_1}{R} \qquad I_2 = \frac{U_2}{R} \\
U_3 &= - \frac{1}{C} \int I \d t = - \frac{1}{C} \int I_1 + I_2 \d t \\
U_R &= \underbrace{ - \frac{1}{RC}}_{\frac{1}{T}} \int U_1 + U_2 \d t
\end{align*}

\subsection*{d)}

\begin{align*}
T &= RC \\ 
\Leftrightarrow C = \frac{T}{e} = \frac{1}{100 \cdot 10^3} = \unit[10]{\mu F}
\end{align*}


\subsection*{e)}

Aufgabenteil fällt weg, weil outdated















\end{document}