\input{header.tex}


\begin{document}

\maketitle

Dieser Text ist unter dieser \href{http://creativecommons.org/licenses/by-nc-sa/4.0/}{Creative Commons} Lizenz veröffentlicht.

\textcolor{red}{Ich erhebe keinen Anspruch auf Vollständigkeit oder Richtigkeit. Falls ihr Fehler findet oder etwas fehlt, dann meldet euch bitte über den Emailkontakt.}

\tableofcontents


\newpage



\section{Aufgabe 1}

Die richtigen Antworten sind:


\begin{center}
	\begin{tabular}{c|c|c|c|c|c|c|c}
		1 & 2 & 3 & 4 & 5 & 6 & 7 & 8 \\ 
		\hline a,c & b & c & a & b & b,c & b & c \\  
	\end{tabular} 
\end{center}


\section{Aufgabe 2}


\subsection*{a)}

\begin{align*}
f = \frac{1}{T} = \frac{1}{\unit[4]{ms}} = \unit[250]{Hz}
\end{align*}


\subsection*{b)}

\begin{align*}
I^2 &= \frac{1}{4} \left[ 20^2 \cdot 2 + (-40)^2 \cdot 2 \right] = \unit[1000]{\mu A^2} \\
I &= \unit[31,6]{\mu A}
\end{align*}


\subsection*{c)}

Man kann aus dem Diagramm erkennen, das $I = \unit[10]{\mu A}$ gilt.


\subsection*{d)}

\paragraph{Schritt 1}

Wechselspannungseinkopplung, das macht den Gleichtaktanteil zu 0.

\paragraph{Schritt 2}

Gleichrichtung

\paragraph{Schritt 3}

arithmetischer Mittelwert $\unit[30]{\mu A}$

\paragraph{Schritt 4}

Multiplikation mit Formfaktor für Sinus

\begin{align*}
A_\sim = 1,11111 \cdot 30 = \unit[33,333]{\mu A}
\end{align*}


\section{Aufgabe 3}

\subsection*{a)}

\begin{figure}[h]
	\centering
	\includegraphics[scale=0.15]{A3_1.jpg}
\end{figure}




\begin{align*}
I = \frac{U_0}{R_Q + R_S} \leq I_K = \frac{U_0}{R_Q}
\intertext{$R_S$ darf nur \unit[1]{\%} von $R_Q$ betragen:}
R_S \leq 0,01 \cdot R_{Q,min} = 0,01 \cdot \unit[0,2]{\Omega} = \unit[0,002]{\Omega}
\end{align*}


\subsection*{b)}

Wir nutzen die Spannungsteilerregel:

\begin{align*}
\frac{U_S}{U_0} &= \frac{R_S}{R_Q + R_S} \\
\Leftrightarrow U_S &= \frac{0,002}{0,3 + 0,002} \cdot 10 = \unit[0,066]{V}
\end{align*}


\subsection*{c)}

\begin{align*}
P &= \frac{U^2}{R} = \frac{0,066^2}{0,002} = \unit[2,19]{W}
\end{align*}

Das ist bei einem so kleinen Bauteil schon eine nicht zu vernachlässigende Verlustleistung.


\subsection*{d)}


Der Shuntwiderstand wird durch die Leistung $P$ erwärmt. Da $R$ von der Temperatur $T$ abhängt kann es zu einer Fehlmessung kommen.


\section{Aufgabe 4}

\subsection*{a)}

\subsubsection*{Aktiv}

\begin{align*}
\frac{U_2}{U_1} = \frac{R_2}{R_1} \cdot \frac{1}{1 + j \omega R_2 C}
\end{align*}


\subsubsection*{Passiv}

\begin{align*}
\frac{U_2}{U_1} = \frac{\frac{1}{j \omega C_2}}{R_2 + \frac{1}{j \omega C_2}} = \frac{1}{1 + j \omega R_2 C_2}
\end{align*}


\subsection*{b)}

\begin{align*}
V &= 10^{\frac{26}{20}} = 19,95 \approx 20 \\
V &= \frac{R_1}{R_2} \\
\Leftrightarrow R_2 &= 19,95 \cdot R_1 = \unit[199,5]{k \Omega}
\end{align*}


\subsection*{c)}

\begin{align*}
T &= R_2 \cdot C_2 = \frac{1}{\omega} = \frac{1}{2 \pi f} \\
\Leftrightarrow C_2 &= \frac{1}{2 \pi f \cdot R_2} = \frac{1}{2 \pi \cdot 100 \cdot 199,5 \cdot 10^3} = \unit[7,98]{nF}
\end{align*}


\subsection*{d)}

\begin{itemize}
	\item Verstärkung
	\item Eingangsimpedanz ist frequenzabhängig
	\item Ausgangimpedanz ist frequenzabhängig
\end{itemize}



















\end{document}